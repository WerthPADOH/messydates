\documentclass[
]{jss}

%% recommended packages
\usepackage{orcidlink,thumbpdf,lmodern}

\usepackage[utf8]{inputenc}

\author{
James Hollway\\Graduate Institute of\\
International and Development Studies \And Henrique Sposito\\Graduate
Institute of\\
International and Development Studies
}
\title{Working with Missing, Approximate, Uncertain, Sets and Ranges of
Dates with \pkg{messydates}}

\Plainauthor{James Hollway, Henrique Sposito}
\Plaintitle{Working with Unspecified, Approximate, Uncertain, Sets and
Ranges of Dates with messydates}
\Shorttitle{\pkg{messydates}: An R package for ISO's Extended Date/Time
Format}


\Abstract{
This paper presents the \pkg{messydates} package for R, which
facilitates working with `messy' dates. Messy dates are common when
studying historical and sometimes even current phenomena, and can create
various technical problems for the data analyst. The paper highlights
these problems and offers practical advice on how to solve them using
\pkg{messydates}. The paper also introduces a conceptual framework for
resolving messydates into more familiar date classes in R ready for
analysis.
}

\Keywords{dates, ISO, \proglang{R}}
\Plainkeywords{dates, ISO, R}

%% publication information
%% \Volume{50}
%% \Issue{9}
%% \Month{June}
%% \Year{2012}
%% \Submitdate{}
%% \Acceptdate{2012-06-04}

\Address{
    James Hollway\\
    Graduate Institute of\\
International and Development Studies\\
    Chemin Eugène-Rigot 2A\\
PO Box 1672\\
1211 Geneva 1\\
Switzerland\\
  E-mail: \email{james.hollway@graduateinstitute.ch}\\
  URL: \url{http://jameshollway.com}\\~\\
    }


% tightlist command for lists without linebreak
\providecommand{\tightlist}{%
  \setlength{\itemsep}{0pt}\setlength{\parskip}{0pt}}




\usepackage{amsmath}

\begin{document}



\hypertarget{introduction}{%
\section{Introduction}\label{introduction}}

Dates are often messy. Whether historical (or ancient), future, or even
recent, we often only know approximately when an event occurred, that it
happened within a particular period, an unreliable source means a date
should be flagged as uncertain, or sources offer multiple, competing
dates.

\pkg{messydates} implements the extended annotation standard for dates,
the Extended Date/Time Format (EDTF), outlined in
\href{https://www.iso.org/standard/70908.html}{ISO 8601-2\_2019(E)} for
R. These include standardised notation for:

\begin{itemize}
\tightlist
\item
  unspecified date( component)s, e.g.~\texttt{2012-XX-01} for the first
  of some unknown month in 2012 or 2012-01 for some unknown day in
  January 2012
\item
  approximate date( component)s,
  e.g.~\texttt{2012-01-12\textasciitilde{}} for approximately the 12th
  of January 2012
\item
  uncertain date( component)s, e.g.~\texttt{2012-01-12?} where this data
  point is based on an unreliable source
\item
  sets of dates, e.g.~\texttt{\{2012-01-01,2012-01-12\}} where the date
  can be both 1 January 2012 and 12 January 2012
\item
  ranges of dates, e.g.~\texttt{2012-01-01..2012-01-12} for all dates
  between the 1 January 2012 and 12 January 2012 inclusive
\end{itemize}

\pkg{messydates} contains a set of tools for constructing and coercing
into and from the \texttt{mdate} class. This date class allows regular
dates to be annotated to express unspecified date components,
approximate or uncertain date components, date ranges, and sets of
dates.

Importantly, the package also includes a function for unpacking or
expanding sets or ranges of dates into all dates consistent with how the
date or set of dates is specified or annotated. Methods are also offered
that can be used to make explicit how researchers convert date
imprecision into precise dates for analysis, such as getting the
\texttt{min()}, \texttt{max()}, or even a \texttt{random()} date from
among the dates consistent with a set or range of dates. This greatly
facilitates research transparency as well as robustness checks.

\hypertarget{motivation}{%
\subsection{Motivation}\label{motivation}}

As researchers, we often recognize this messiness but are forced to
force non-existent precision on data so we can proceed with analysis.
For example, if we only know something happened in a given month or
year, we might just opt for the start of that month
(e.g.~\texttt{2021-07-01}) or year (\texttt{2021-01-01}), assuming that
to err on the earlier (or later) side is a justifiable bias. However,
this can create issues for inference in which sequence or timing is
important. The goal of \pkg{messydates} is to help with this problem by
retaining and working with various kinds of date imprecision.

\hypertarget{relationship-to-other-packages}{%
\subsection{Relationship to other
packages}\label{relationship-to-other-packages}}

\pkg{messydates} offers a new date class, but one that comes with
methods for converting from and into common date classes such as
\texttt{Date}, \texttt{POSIXct}, and \texttt{POSIXlt.} It is thus fully
compatible with packages such as \pkg{lubridate}
\citep{grolemundDatesTimesMade2011} and \pkg{anytime}
\citep{eddelbuettelAnytimeEasierDate2019}.

\section[R code]{\proglang{R} code}\label{r-code}

\hypertarget{a-new-class}{%
\subsection{A new class}\label{a-new-class}}

\pkg{messydates} contains a set of tools for constructing and coercing
into and from the \texttt{mdate} class. This date class implements ISO
8601-2:2019(E) and allows regular dates to be annotated to express
unspecified date components, approximate or uncertain date components,
date ranges, and sets of dates. The function \texttt{as\_messydate()}
handles the coercion to \texttt{mdate} class.

\begin{CodeChunk}
\begin{CodeInput}
R> library(messydates)
R> suppressPackageStartupMessages(library(lubridate))
R> suppressPackageStartupMessages(library(anytime))
R> library(tibble)
R> suppressPackageStartupMessages(library(dplyr))
R> dates_comparison <- tibble::tribble(~Example, ~OriginalDate,
+                                     "A normal date", as.character(Sys.Date()),
+                                     "A future date", "2599-12-31",
+                                     "A written date", "First of February, two thousand and twenty-one",
+                                     "A historical date", "476",
+                                     "An era date", "33 BC",
+                                     "An approximate date", "2012-01-12~",
+                                     "An uncertain date", "2001-01-01?",
+                                     "An unspecified date", "2012-01",
+                                     "A censored date", "..2012-01-12", 
+                                     "A range of dates", "2019-11-01:2020-01-01",
+                                     "A set of dates", "2021-5-26, 2021-11-19, 2021-12-4") %>%
+   dplyr::mutate(base = as.Date(OriginalDate),
+                 lubridate = lubridate::as_date(OriginalDate),
+                 anytime = anytime::anydate(OriginalDate),
+                 messydates = messydates::as_messydate(OriginalDate)) %>%
+   print()
\end{CodeInput}
\begin{CodeOutput}
# A tibble: 11 x 6
   Example             OriginalDate  base       lubridate  anytime    messydates
   <chr>               <chr>         <date>     <date>     <date>     <mdate>   
 1 A normal date       2022-06-15    2022-06-15 2022-06-15 2022-06-15 2022-06-1~
 2 A future date       2599-12-31    2599-12-31 2599-12-31 2599-12-31 2599-12-3~
 3 A written date      First of Feb~ NA         NA         NA         2021-02-0~
 4 A historical date   476           NA         NA         NA         0476     ~
 5 An era date         33 BC         NA         NA         NA         -0033    ~
 6 An approximate date 2012-01-12~   2012-01-12 2012-01-12 2012-01-12 2012-01-1~
 7 An uncertain date   2001-01-01?   2001-01-01 2001-01-01 2001-01-01 2001-01-0~
 8 An unspecified date 2012-01       NA         2020-12-01 2012-01-01 2012-01  ~
 9 A censored date     ..2012-01-12  NA         2012-01-12 NA         ..2012-01~
10 A range of dates    2019-11-01:2~ 2019-11-01 2019-11-01 2019-11-01 2019-11-0~
11 A set of dates      2021-5-26, 2~ 2021-05-26 NA         2021-05-26 {2021-05-~
\end{CodeOutput}
\end{CodeChunk}

\hypertarget{annotate}{%
\subsection{Annotate}\label{annotate}}

Some datasets have, for example, an arbitrary cut off point for start
and end points, but these are often coded as precise dates when they are
not necessarily the real start or end dates. The annotate functions
helps annotate uncertainty and approximation to dates. Inaccurate start
or end dates can be represented by an affix indicating ``on or before'',
if used as a prefix (e.g.~\texttt{..1816-01-01}), or indicating ``on or
after'', if used as a suffix (e.g.~\texttt{2016-12-31..}). Approximate
dates are indicated by adding a \texttt{\textasciitilde{}} to year,
month, or day components, as well as groups of components or whole dates
to estimate values that are possibly correct
(e.g.~\texttt{2003-03-03\textasciitilde{}}). Day, month, or year,
uncertainty can be indicated by adding a \texttt{?} to a possibly
dubious date (e.g.~\texttt{1916-10-10?}) or date component
(e.g.~\texttt{1916-?10-10}).

\begin{CodeChunk}
\begin{CodeInput}
R> dates_annotate <- tibble::tibble(Beg = as_messydate(c("1816-01-01", "1916-01-01", "2016-01-01")),
+                                  End = as_messydate(c("1816-12-31", "1916-12-31", "2016-12-31")))
R> dplyr::mutate(dates_annotate, Beg = ifelse(Beg <= "1816-01-01", on_or_before(Beg), Beg))
\end{CodeInput}
\begin{CodeOutput}
# A tibble: 3 x 2
  Beg          End       
  <chr>        <mdate>   
1 ..1816-01-01 1816-12-31
2 1916-01-01   1916-12-31
3 2016-01-01   2016-12-31
\end{CodeOutput}
\begin{CodeInput}
R> dplyr::mutate(dates_annotate, End = ifelse(End >= "2016-01-01", on_or_after(End), End))
\end{CodeInput}
\begin{CodeOutput}
# A tibble: 3 x 2
  Beg        End         
  <mdate>    <chr>       
1 1816-01-01 1816-12-31  
2 1916-01-01 1916-12-31  
3 2016-01-01 2016-12-31..
\end{CodeOutput}
\begin{CodeInput}
R> dplyr::mutate(dates_annotate, Beg = ifelse(Beg == "1916-01-01", as_approximate(Beg), Beg))
\end{CodeInput}
\begin{CodeOutput}
# A tibble: 3 x 2
  Beg         End       
  <chr>       <mdate>   
1 1816-01-01  1816-12-31
2 1916-01-01~ 1916-12-31
3 2016-01-01  2016-12-31
\end{CodeOutput}
\begin{CodeInput}
R> dplyr::mutate(dates_annotate, End = ifelse(End == "1916-12-31", as_uncertain(End), End))
\end{CodeInput}
\begin{CodeOutput}
# A tibble: 3 x 2
  Beg        End        
  <mdate>    <chr>      
1 1816-01-01 1816-12-31 
2 1916-01-01 1916-12-31?
3 2016-01-01 2016-12-31 
\end{CodeOutput}
\end{CodeChunk}

\hypertarget{expand}{%
\subsection{Expand}\label{expand}}

Expand functions transform date ranges, sets of dates, and unspecified
or approximate dates (annotated with `..', `\{ , \}', `XX' or
`\textasciitilde{}') into lists of dates. As these dates may refer to
several possible dates, the function ``opens'' these values to include
all the possible dates implied.

\begin{CodeChunk}
\begin{CodeInput}
R> dates_expand <- as_messydate(c("2001-01-01", "2001-01?", "2001",
+                                "2001-01-01..2001-02-02", "{2001-01-01,2001-02-02}",
+                                "2001-XX-31", "21 BC"))
R> expand(dates_expand)
\end{CodeInput}
\begin{CodeOutput}
Please specify 'approx_range' argument if you want approximate dates to also be expanded
\end{CodeOutput}
\begin{CodeOutput}
[[1]]
[1] "2001-01-01"

[[2]]
 [1] "2001-01-01" "2001-01-02" "2001-01-03" "2001-01-04" "2001-01-05"
 [6] "2001-01-06" "2001-01-07" "2001-01-08" "2001-01-09" "2001-01-10"
[11] "2001-01-11" "2001-01-12" "2001-01-13" "2001-01-14" "2001-01-15"
[16] "2001-01-16" "2001-01-17" "2001-01-18" "2001-01-19" "2001-01-20"
[21] "2001-01-21" "2001-01-22" "2001-01-23" "2001-01-24" "2001-01-25"
[26] "2001-01-26" "2001-01-27" "2001-01-28" "2001-01-29" "2001-01-30"
[31] "2001-01-31"

[[3]]
  [1] "2001-01-01" "2001-01-02" "2001-01-03" "2001-01-04" "2001-01-05"
  [6] "2001-01-06" "2001-01-07" "2001-01-08" "2001-01-09" "2001-01-10"
 [11] "2001-01-11" "2001-01-12" "2001-01-13" "2001-01-14" "2001-01-15"
 [16] "2001-01-16" "2001-01-17" "2001-01-18" "2001-01-19" "2001-01-20"
 [21] "2001-01-21" "2001-01-22" "2001-01-23" "2001-01-24" "2001-01-25"
 [26] "2001-01-26" "2001-01-27" "2001-01-28" "2001-01-29" "2001-01-30"
 [31] "2001-01-31" "2001-02-01" "2001-02-02" "2001-02-03" "2001-02-04"
 [36] "2001-02-05" "2001-02-06" "2001-02-07" "2001-02-08" "2001-02-09"
 [41] "2001-02-10" "2001-02-11" "2001-02-12" "2001-02-13" "2001-02-14"
 [46] "2001-02-15" "2001-02-16" "2001-02-17" "2001-02-18" "2001-02-19"
 [51] "2001-02-20" "2001-02-21" "2001-02-22" "2001-02-23" "2001-02-24"
 [56] "2001-02-25" "2001-02-26" "2001-02-27" "2001-02-28" "2001-03-01"
 [61] "2001-03-02" "2001-03-03" "2001-03-04" "2001-03-05" "2001-03-06"
 [66] "2001-03-07" "2001-03-08" "2001-03-09" "2001-03-10" "2001-03-11"
 [71] "2001-03-12" "2001-03-13" "2001-03-14" "2001-03-15" "2001-03-16"
 [76] "2001-03-17" "2001-03-18" "2001-03-19" "2001-03-20" "2001-03-21"
 [81] "2001-03-22" "2001-03-23" "2001-03-24" "2001-03-25" "2001-03-26"
 [86] "2001-03-27" "2001-03-28" "2001-03-29" "2001-03-30" "2001-03-31"
 [91] "2001-04-01" "2001-04-02" "2001-04-03" "2001-04-04" "2001-04-05"
 [96] "2001-04-06" "2001-04-07" "2001-04-08" "2001-04-09" "2001-04-10"
[101] "2001-04-11" "2001-04-12" "2001-04-13" "2001-04-14" "2001-04-15"
[106] "2001-04-16" "2001-04-17" "2001-04-18" "2001-04-19" "2001-04-20"
[111] "2001-04-21" "2001-04-22" "2001-04-23" "2001-04-24" "2001-04-25"
[116] "2001-04-26" "2001-04-27" "2001-04-28" "2001-04-29" "2001-04-30"
[121] "2001-05-01" "2001-05-02" "2001-05-03" "2001-05-04" "2001-05-05"
[126] "2001-05-06" "2001-05-07" "2001-05-08" "2001-05-09" "2001-05-10"
[131] "2001-05-11" "2001-05-12" "2001-05-13" "2001-05-14" "2001-05-15"
[136] "2001-05-16" "2001-05-17" "2001-05-18" "2001-05-19" "2001-05-20"
[141] "2001-05-21" "2001-05-22" "2001-05-23" "2001-05-24" "2001-05-25"
[146] "2001-05-26" "2001-05-27" "2001-05-28" "2001-05-29" "2001-05-30"
[151] "2001-05-31" "2001-06-01" "2001-06-02" "2001-06-03" "2001-06-04"
[156] "2001-06-05" "2001-06-06" "2001-06-07" "2001-06-08" "2001-06-09"
[161] "2001-06-10" "2001-06-11" "2001-06-12" "2001-06-13" "2001-06-14"
[166] "2001-06-15" "2001-06-16" "2001-06-17" "2001-06-18" "2001-06-19"
[171] "2001-06-20" "2001-06-21" "2001-06-22" "2001-06-23" "2001-06-24"
[176] "2001-06-25" "2001-06-26" "2001-06-27" "2001-06-28" "2001-06-29"
[181] "2001-06-30" "2001-07-01" "2001-07-02" "2001-07-03" "2001-07-04"
[186] "2001-07-05" "2001-07-06" "2001-07-07" "2001-07-08" "2001-07-09"
[191] "2001-07-10" "2001-07-11" "2001-07-12" "2001-07-13" "2001-07-14"
[196] "2001-07-15" "2001-07-16" "2001-07-17" "2001-07-18" "2001-07-19"
[201] "2001-07-20" "2001-07-21" "2001-07-22" "2001-07-23" "2001-07-24"
[206] "2001-07-25" "2001-07-26" "2001-07-27" "2001-07-28" "2001-07-29"
[211] "2001-07-30" "2001-07-31" "2001-08-01" "2001-08-02" "2001-08-03"
[216] "2001-08-04" "2001-08-05" "2001-08-06" "2001-08-07" "2001-08-08"
[221] "2001-08-09" "2001-08-10" "2001-08-11" "2001-08-12" "2001-08-13"
[226] "2001-08-14" "2001-08-15" "2001-08-16" "2001-08-17" "2001-08-18"
[231] "2001-08-19" "2001-08-20" "2001-08-21" "2001-08-22" "2001-08-23"
[236] "2001-08-24" "2001-08-25" "2001-08-26" "2001-08-27" "2001-08-28"
[241] "2001-08-29" "2001-08-30" "2001-08-31" "2001-09-01" "2001-09-02"
[246] "2001-09-03" "2001-09-04" "2001-09-05" "2001-09-06" "2001-09-07"
[251] "2001-09-08" "2001-09-09" "2001-09-10" "2001-09-11" "2001-09-12"
[256] "2001-09-13" "2001-09-14" "2001-09-15" "2001-09-16" "2001-09-17"
[261] "2001-09-18" "2001-09-19" "2001-09-20" "2001-09-21" "2001-09-22"
[266] "2001-09-23" "2001-09-24" "2001-09-25" "2001-09-26" "2001-09-27"
[271] "2001-09-28" "2001-09-29" "2001-09-30" "2001-10-01" "2001-10-02"
[276] "2001-10-03" "2001-10-04" "2001-10-05" "2001-10-06" "2001-10-07"
[281] "2001-10-08" "2001-10-09" "2001-10-10" "2001-10-11" "2001-10-12"
[286] "2001-10-13" "2001-10-14" "2001-10-15" "2001-10-16" "2001-10-17"
[291] "2001-10-18" "2001-10-19" "2001-10-20" "2001-10-21" "2001-10-22"
[296] "2001-10-23" "2001-10-24" "2001-10-25" "2001-10-26" "2001-10-27"
[301] "2001-10-28" "2001-10-29" "2001-10-30" "2001-10-31" "2001-11-01"
[306] "2001-11-02" "2001-11-03" "2001-11-04" "2001-11-05" "2001-11-06"
[311] "2001-11-07" "2001-11-08" "2001-11-09" "2001-11-10" "2001-11-11"
[316] "2001-11-12" "2001-11-13" "2001-11-14" "2001-11-15" "2001-11-16"
[321] "2001-11-17" "2001-11-18" "2001-11-19" "2001-11-20" "2001-11-21"
[326] "2001-11-22" "2001-11-23" "2001-11-24" "2001-11-25" "2001-11-26"
[331] "2001-11-27" "2001-11-28" "2001-11-29" "2001-11-30" "2001-12-01"
[336] "2001-12-02" "2001-12-03" "2001-12-04" "2001-12-05" "2001-12-06"
[341] "2001-12-07" "2001-12-08" "2001-12-09" "2001-12-10" "2001-12-11"
[346] "2001-12-12" "2001-12-13" "2001-12-14" "2001-12-15" "2001-12-16"
[351] "2001-12-17" "2001-12-18" "2001-12-19" "2001-12-20" "2001-12-21"
[356] "2001-12-22" "2001-12-23" "2001-12-24" "2001-12-25" "2001-12-26"
[361] "2001-12-27" "2001-12-28" "2001-12-29" "2001-12-30" "2001-12-31"

[[4]]
 [1] "2001-01-01" "2001-01-02" "2001-01-03" "2001-01-04" "2001-01-05"
 [6] "2001-01-06" "2001-01-07" "2001-01-08" "2001-01-09" "2001-01-10"
[11] "2001-01-11" "2001-01-12" "2001-01-13" "2001-01-14" "2001-01-15"
[16] "2001-01-16" "2001-01-17" "2001-01-18" "2001-01-19" "2001-01-20"
[21] "2001-01-21" "2001-01-22" "2001-01-23" "2001-01-24" "2001-01-25"
[26] "2001-01-26" "2001-01-27" "2001-01-28" "2001-01-29" "2001-01-30"
[31] "2001-01-31" "2001-02-01" "2001-02-02"

[[5]]
[1] "2001-01-01" "2001-02-02"

[[6]]
 [1] "2001-01-31" "2001-02-28" "2001-03-31" "2001-04-30" "2001-05-31"
 [6] "2001-06-30" "2001-07-31" "2001-08-31" "2001-09-30" "2001-10-31"
[11] "2001-11-30" "2001-12-31"

[[7]]
  [1] "-0021-01-01" "-0021-01-02" "-0021-01-03" "-0021-01-04" "-0021-01-05"
  [6] "-0021-01-06" "-0021-01-07" "-0021-01-08" "-0021-01-09" "-0021-01-10"
 [11] "-0021-01-11" "-0021-01-12" "-0021-01-13" "-0021-01-14" "-0021-01-15"
 [16] "-0021-01-16" "-0021-01-17" "-0021-01-18" "-0021-01-19" "-0021-01-20"
 [21] "-0021-01-21" "-0021-01-22" "-0021-01-23" "-0021-01-24" "-0021-01-25"
 [26] "-0021-01-26" "-0021-01-27" "-0021-01-28" "-0021-01-29" "-0021-01-30"
 [31] "-0021-01-31" "-0021-02-01" "-0021-02-02" "-0021-02-03" "-0021-02-04"
 [36] "-0021-02-05" "-0021-02-06" "-0021-02-07" "-0021-02-08" "-0021-02-09"
 [41] "-0021-02-10" "-0021-02-11" "-0021-02-12" "-0021-02-13" "-0021-02-14"
 [46] "-0021-02-15" "-0021-02-16" "-0021-02-17" "-0021-02-18" "-0021-02-19"
 [51] "-0021-02-20" "-0021-02-21" "-0021-02-22" "-0021-02-23" "-0021-02-24"
 [56] "-0021-02-25" "-0021-02-26" "-0021-02-27" "-0021-02-28" "-0021-03-01"
 [61] "-0021-03-02" "-0021-03-03" "-0021-03-04" "-0021-03-05" "-0021-03-06"
 [66] "-0021-03-07" "-0021-03-08" "-0021-03-09" "-0021-03-10" "-0021-03-11"
 [71] "-0021-03-12" "-0021-03-13" "-0021-03-14" "-0021-03-15" "-0021-03-16"
 [76] "-0021-03-17" "-0021-03-18" "-0021-03-19" "-0021-03-20" "-0021-03-21"
 [81] "-0021-03-22" "-0021-03-23" "-0021-03-24" "-0021-03-25" "-0021-03-26"
 [86] "-0021-03-27" "-0021-03-28" "-0021-03-29" "-0021-03-30" "-0021-03-31"
 [91] "-0021-04-01" "-0021-04-02" "-0021-04-03" "-0021-04-04" "-0021-04-05"
 [96] "-0021-04-06" "-0021-04-07" "-0021-04-08" "-0021-04-09" "-0021-04-10"
[101] "-0021-04-11" "-0021-04-12" "-0021-04-13" "-0021-04-14" "-0021-04-15"
[106] "-0021-04-16" "-0021-04-17" "-0021-04-18" "-0021-04-19" "-0021-04-20"
[111] "-0021-04-21" "-0021-04-22" "-0021-04-23" "-0021-04-24" "-0021-04-25"
[116] "-0021-04-26" "-0021-04-27" "-0021-04-28" "-0021-04-29" "-0021-04-30"
[121] "-0021-05-01" "-0021-05-02" "-0021-05-03" "-0021-05-04" "-0021-05-05"
[126] "-0021-05-06" "-0021-05-07" "-0021-05-08" "-0021-05-09" "-0021-05-10"
[131] "-0021-05-11" "-0021-05-12" "-0021-05-13" "-0021-05-14" "-0021-05-15"
[136] "-0021-05-16" "-0021-05-17" "-0021-05-18" "-0021-05-19" "-0021-05-20"
[141] "-0021-05-21" "-0021-05-22" "-0021-05-23" "-0021-05-24" "-0021-05-25"
[146] "-0021-05-26" "-0021-05-27" "-0021-05-28" "-0021-05-29" "-0021-05-30"
[151] "-0021-05-31" "-0021-06-01" "-0021-06-02" "-0021-06-03" "-0021-06-04"
[156] "-0021-06-05" "-0021-06-06" "-0021-06-07" "-0021-06-08" "-0021-06-09"
[161] "-0021-06-10" "-0021-06-11" "-0021-06-12" "-0021-06-13" "-0021-06-14"
[166] "-0021-06-15" "-0021-06-16" "-0021-06-17" "-0021-06-18" "-0021-06-19"
[171] "-0021-06-20" "-0021-06-21" "-0021-06-22" "-0021-06-23" "-0021-06-24"
[176] "-0021-06-25" "-0021-06-26" "-0021-06-27" "-0021-06-28" "-0021-06-29"
[181] "-0021-06-30" "-0021-07-01" "-0021-07-02" "-0021-07-03" "-0021-07-04"
[186] "-0021-07-05" "-0021-07-06" "-0021-07-07" "-0021-07-08" "-0021-07-09"
[191] "-0021-07-10" "-0021-07-11" "-0021-07-12" "-0021-07-13" "-0021-07-14"
[196] "-0021-07-15" "-0021-07-16" "-0021-07-17" "-0021-07-18" "-0021-07-19"
[201] "-0021-07-20" "-0021-07-21" "-0021-07-22" "-0021-07-23" "-0021-07-24"
[206] "-0021-07-25" "-0021-07-26" "-0021-07-27" "-0021-07-28" "-0021-07-29"
[211] "-0021-07-30" "-0021-07-31" "-0021-08-01" "-0021-08-02" "-0021-08-03"
[216] "-0021-08-04" "-0021-08-05" "-0021-08-06" "-0021-08-07" "-0021-08-08"
[221] "-0021-08-09" "-0021-08-10" "-0021-08-11" "-0021-08-12" "-0021-08-13"
[226] "-0021-08-14" "-0021-08-15" "-0021-08-16" "-0021-08-17" "-0021-08-18"
[231] "-0021-08-19" "-0021-08-20" "-0021-08-21" "-0021-08-22" "-0021-08-23"
[236] "-0021-08-24" "-0021-08-25" "-0021-08-26" "-0021-08-27" "-0021-08-28"
[241] "-0021-08-29" "-0021-08-30" "-0021-08-31" "-0021-09-01" "-0021-09-02"
[246] "-0021-09-03" "-0021-09-04" "-0021-09-05" "-0021-09-06" "-0021-09-07"
[251] "-0021-09-08" "-0021-09-09" "-0021-09-10" "-0021-09-11" "-0021-09-12"
[256] "-0021-09-13" "-0021-09-14" "-0021-09-15" "-0021-09-16" "-0021-09-17"
[261] "-0021-09-18" "-0021-09-19" "-0021-09-20" "-0021-09-21" "-0021-09-22"
[266] "-0021-09-23" "-0021-09-24" "-0021-09-25" "-0021-09-26" "-0021-09-27"
[271] "-0021-09-28" "-0021-09-29" "-0021-09-30" "-0021-10-01" "-0021-10-02"
[276] "-0021-10-03" "-0021-10-04" "-0021-10-05" "-0021-10-06" "-0021-10-07"
[281] "-0021-10-08" "-0021-10-09" "-0021-10-10" "-0021-10-11" "-0021-10-12"
[286] "-0021-10-13" "-0021-10-14" "-0021-10-15" "-0021-10-16" "-0021-10-17"
[291] "-0021-10-18" "-0021-10-19" "-0021-10-20" "-0021-10-21" "-0021-10-22"
[296] "-0021-10-23" "-0021-10-24" "-0021-10-25" "-0021-10-26" "-0021-10-27"
[301] "-0021-10-28" "-0021-10-29" "-0021-10-30" "-0021-10-31" "-0021-11-01"
[306] "-0021-11-02" "-0021-11-03" "-0021-11-04" "-0021-11-05" "-0021-11-06"
[311] "-0021-11-07" "-0021-11-08" "-0021-11-09" "-0021-11-10" "-0021-11-11"
[316] "-0021-11-12" "-0021-11-13" "-0021-11-14" "-0021-11-15" "-0021-11-16"
[321] "-0021-11-17" "-0021-11-18" "-0021-11-19" "-0021-11-20" "-0021-11-21"
[326] "-0021-11-22" "-0021-11-23" "-0021-11-24" "-0021-11-25" "-0021-11-26"
[331] "-0021-11-27" "-0021-11-28" "-0021-11-29" "-0021-11-30" "-0021-12-01"
[336] "-0021-12-02" "-0021-12-03" "-0021-12-04" "-0021-12-05" "-0021-12-06"
[341] "-0021-12-07" "-0021-12-08" "-0021-12-09" "-0021-12-10" "-0021-12-11"
[346] "-0021-12-12" "-0021-12-13" "-0021-12-14" "-0021-12-15" "-0021-12-16"
[351] "-0021-12-17" "-0021-12-18" "-0021-12-19" "-0021-12-20" "-0021-12-21"
[356] "-0021-12-22" "-0021-12-23" "-0021-12-24" "-0021-12-25" "-0021-12-26"
[361] "-0021-12-27" "-0021-12-28" "-0021-12-29" "-0021-12-30" "-0021-12-31"
\end{CodeOutput}
\end{CodeChunk}

\hypertarget{contract}{%
\subsection{Contract}\label{contract}}

The \texttt{contract()} function operates as the opposite of
\texttt{expand()}. It contracts a list of dates into the abbreviated
annotation of \pkg{messydates}.

\begin{CodeChunk}
\begin{CodeInput}
R> tibble::tibble(contract = contract(expand(dates_expand)))
\end{CodeInput}
\begin{CodeOutput}
# A tibble: 7 x 1
  contract                                                                      
  <mdate>                                                                       
1 2001-01-01                                                                   ~
2 2001-01                                                                      ~
3 2001                                                                         ~
4 2001-01-01..2001-02-02                                                       ~
5 {2001-01-01,2001-02-02}                                                      ~
6 {2001-01-31,2001-02-28,2001-03-31,2001-04-30,2001-05-31,2001-06-30,2001-07-31~
7 -0021                                                                        ~
\end{CodeOutput}
\end{CodeChunk}

\hypertarget{coerce-from-messydates}{%
\subsection{Coerce from messydates}\label{coerce-from-messydates}}

Coercion functions coerce objects of \texttt{mdate} class to common date
classes such as \texttt{Date}, \texttt{POSIXct}, and \texttt{POSIXlt}.
Since \texttt{mdate} objects can hold multiple individual dates, an
additional function must be passed as an argument so that multiple dates
are ``resolved'' into a single date.

For example, one might wish to use the earliest possible date in any
ranges of dates (\texttt{min}), the latest possible date (\texttt{max}),
some notion of a central tendency (\texttt{mean}, \texttt{median}, or
\texttt{modal}), or even a \texttt{random} selection from amongst the
candidate dates.

These functions are particularly useful for use with existing methods
and models, especially for checking the robustness of results.

\begin{CodeChunk}
\begin{CodeInput}
R> tibble::tibble(min = as.Date(dates_expand, min),
+                max = as.Date(dates_expand, max),
+                median = as.Date(dates_expand, median),
+                mean = as.Date(dates_expand, mean),
+                modal = as.Date(dates_expand, modal),
+                random = as.Date(dates_expand, random))
\end{CodeInput}
\begin{CodeOutput}
# A tibble: 7 x 6
  min        max        median     mean       modal      random    
  <date>     <date>     <date>     <date>     <date>     <date>    
1 2001-01-01 2001-01-01 2001-01-01 2001-01-01 2001-01-01 2001-01-01
2 2001-01-01 2001-01-31 2001-01-16 2001-01-16 2001-01-01 2001-01-26
3 2001-01-01 2001-12-31 2001-07-02 2001-07-02 2001-01-01 2001-11-13
4 2001-01-01 2001-02-02 2001-01-17 2001-01-17 2001-01-01 2001-01-12
5 2001-01-01 2001-02-02 2001-02-02 2001-01-17 2001-01-01 2001-01-01
6 2001-01-31 2001-12-31 2001-07-31 2001-07-15 2001-01-31 2001-12-31
7 -021-01-01 -021-12-31 -021-07-02 -021-07-02 -021-01-01 -021-06-15
\end{CodeOutput}
\end{CodeChunk}

\hypertarget{additional-functionality}{%
\subsection{Additional functionality}\label{additional-functionality}}

Several other functions are also offered in the \pkg{messydates}
package.

For example, one can check various logical tests for messy date objects.
\texttt{is\_messydate()} tests whether the object inherits the
\texttt{mdate} class. \texttt{is\_intersecting()} tests whether there is
any intersection between two messy dates. \texttt{is\_element()}
similarly tests whether a messy date can be found within a messy date
range or set. \texttt{is\_similar()} tests whether two dates contain
similar components. \texttt{is\_precise()} tests for whether date is
precise.

\begin{CodeChunk}
\begin{CodeInput}
R> is_messydate(as_messydate("2001-01-01"))
\end{CodeInput}
\begin{CodeOutput}
[1] TRUE
\end{CodeOutput}
\begin{CodeInput}
R> is_messydate(as.Date("2001-01-01"))
\end{CodeInput}
\begin{CodeOutput}
[1] FALSE
\end{CodeOutput}
\begin{CodeInput}
R> is_intersecting(as_messydate("2001-01"), as_messydate("2001-01-01..2001-02-22"))
\end{CodeInput}
\begin{CodeOutput}
Please specify 'approx_range' argument if you want approximate dates to also be expanded
Please specify 'approx_range' argument if you want approximate dates to also be expanded
\end{CodeOutput}
\begin{CodeOutput}
[1] TRUE
\end{CodeOutput}
\begin{CodeInput}
R> is_intersecting(as_messydate("2001-01"), as_messydate("2001-02-01..2001-02-22"))
\end{CodeInput}
\begin{CodeOutput}
Please specify 'approx_range' argument if you want approximate dates to also be expanded
Please specify 'approx_range' argument if you want approximate dates to also be expanded
\end{CodeOutput}
\begin{CodeOutput}
[1] FALSE
\end{CodeOutput}
\begin{CodeInput}
R> is_element(as_messydate("2001-01-01"), as_messydate("2001-01"))
\end{CodeInput}
\begin{CodeOutput}
Please specify 'approx_range' argument if you want approximate dates to also be expanded
\end{CodeOutput}
\begin{CodeOutput}
[1] TRUE
\end{CodeOutput}
\begin{CodeInput}
R> is_element(as_messydate("2001-01-01"), as_messydate("2001-02"))
\end{CodeInput}
\begin{CodeOutput}
Please specify 'approx_range' argument if you want approximate dates to also be expanded
\end{CodeOutput}
\begin{CodeOutput}
[1] FALSE
\end{CodeOutput}
\begin{CodeInput}
R> is_similar(as_messydate("2001-06-02"), as_messydate("2001-02-06"))
\end{CodeInput}
\begin{CodeOutput}
[1] TRUE
\end{CodeOutput}
\begin{CodeInput}
R> is_similar(as_messydate("2001-06-22"), as_messydate("2001-02-06"))
\end{CodeInput}
\begin{CodeOutput}
[1] FALSE
\end{CodeOutput}
\begin{CodeInput}
R> is_precise(as_messydate("2001-06-02"))
\end{CodeInput}
\begin{CodeOutput}
[1] TRUE
\end{CodeOutput}
\begin{CodeInput}
R> is_precise(as_messydate("2001-02"))
\end{CodeInput}
\begin{CodeOutput}
[1] FALSE
\end{CodeOutput}
\end{CodeChunk}

Additionally, one can perform intersection (\texttt{md\_intersect()})
and union (\texttt{md\_union()}) on, inter alia, messy date class
objects. Or `join' that retains all elements, even if duplicated, with
\texttt{md\_multiset}.

\begin{CodeChunk}
\begin{CodeInput}
R> md_intersect(as_messydate("2001-01-01..2001-01-20"),as_messydate("2001-01"))
\end{CodeInput}
\begin{CodeOutput}
Please specify 'approx_range' argument if you want approximate dates to also be expanded
Please specify 'approx_range' argument if you want approximate dates to also be expanded
\end{CodeOutput}
\begin{CodeOutput}
 [1] "2001-01-01" "2001-01-02" "2001-01-03" "2001-01-04" "2001-01-05"
 [6] "2001-01-06" "2001-01-07" "2001-01-08" "2001-01-09" "2001-01-10"
[11] "2001-01-11" "2001-01-12" "2001-01-13" "2001-01-14" "2001-01-15"
[16] "2001-01-16" "2001-01-17" "2001-01-18" "2001-01-19" "2001-01-20"
\end{CodeOutput}
\begin{CodeInput}
R> md_union(as_messydate("2001-01-01..2001-01-20"),as_messydate("2001-01"))
\end{CodeInput}
\begin{CodeOutput}
Please specify 'approx_range' argument if you want approximate dates to also be expanded
Please specify 'approx_range' argument if you want approximate dates to also be expanded
\end{CodeOutput}
\begin{CodeOutput}
 [1] "2001-01-01" "2001-01-02" "2001-01-03" "2001-01-04" "2001-01-05"
 [6] "2001-01-06" "2001-01-07" "2001-01-08" "2001-01-09" "2001-01-10"
[11] "2001-01-11" "2001-01-12" "2001-01-13" "2001-01-14" "2001-01-15"
[16] "2001-01-16" "2001-01-17" "2001-01-18" "2001-01-19" "2001-01-20"
[21] "2001-01-21" "2001-01-22" "2001-01-23" "2001-01-24" "2001-01-25"
[26] "2001-01-26" "2001-01-27" "2001-01-28" "2001-01-29" "2001-01-30"
[31] "2001-01-31"
\end{CodeOutput}
\begin{CodeInput}
R> md_multiset(as_messydate("2001-01-01..2001-01-20"),as_messydate("2001-01"))
\end{CodeInput}
\begin{CodeOutput}
Please specify 'approx_range' argument if you want approximate dates to also be expanded
Please specify 'approx_range' argument if you want approximate dates to also be expanded
\end{CodeOutput}
\begin{CodeOutput}
 [1] "2001-01-01" "2001-01-02" "2001-01-03" "2001-01-04" "2001-01-05"
 [6] "2001-01-06" "2001-01-07" "2001-01-08" "2001-01-09" "2001-01-10"
[11] "2001-01-11" "2001-01-12" "2001-01-13" "2001-01-14" "2001-01-15"
[16] "2001-01-16" "2001-01-17" "2001-01-18" "2001-01-19" "2001-01-20"
[21] "2001-01-01" "2001-01-02" "2001-01-03" "2001-01-04" "2001-01-05"
[26] "2001-01-06" "2001-01-07" "2001-01-08" "2001-01-09" "2001-01-10"
[31] "2001-01-11" "2001-01-12" "2001-01-13" "2001-01-14" "2001-01-15"
[36] "2001-01-16" "2001-01-17" "2001-01-18" "2001-01-19" "2001-01-20"
[41] "2001-01-21" "2001-01-22" "2001-01-23" "2001-01-24" "2001-01-25"
[46] "2001-01-26" "2001-01-27" "2001-01-28" "2001-01-29" "2001-01-30"
[51] "2001-01-31"
\end{CodeOutput}
\end{CodeChunk}

As well, some arithmetic operations are available for messydates. For
instance, one can add or subtract one year to all messy dates in a
vector.

\begin{CodeChunk}
\begin{CodeInput}
R> tibble::tibble(date = dates_expand, add = dates_expand + "1 day", subtract = dates_expand - "1 year")
\end{CodeInput}
\begin{CodeOutput}
# A tibble: 7 x 3
  date                    add                                           subtract
  <mdate>                 <mdate>                                       <mdate> 
1 2001-01-01              2001-01-02                                  ~ 2000-01~
2 2001-01?                2001-01-02..2001-02-01                      ~ 2000-01~
3 2001                    2001-01-02..2002-01-01                      ~ 2000-01~
4 2001-01-01..2001-02-02  2001-01-02..2001-02-03                      ~ 2000-01~
5 {2001-01-01,2001-02-02} {2001-01-02,2001-02-03}                     ~ {2000-0~
6 2001-XX-31              {2001-02-01,2001-03-01,2001-04-01,2001-05-01~ {2000-0~
7 -0021                   -0020-12-31..-0021-12-30                    ~ -0022  ~
\end{CodeOutput}
\end{CodeChunk}

\hypertarget{case-study---2001-battles}{%
\subsection{Case Study - 2001 Battles}\label{case-study---2001-battles}}

Dates, even for some recent events, can be messy. Take, for example, the
dates of battles in 2001 according to
\href{https://en.wikipedia.org/wiki/List_of_battles_in_the_21st_century}{Wikipedia}.
The dates of these battles are often uncertain, with different levels of
date precision being reported. \pkg{messydates} facilitates working with
these dates.

\begin{CodeChunk}
\begin{CodeInput}
R> battles <- tibble::tribble(~Battle, ~Date, ~Parties,
+                            "Operation MH-2", "2001 March 8", "MK-National Libration Army(MK)",
+                            "2001 Bangladesh–India border clashes", "2001-04-16..2001-04-20", "BD-ID",
+                            "Operation Vaksince", "25-5-2001", "MK-National Libration Army(MK)",
+                            "Alkhan-Kala operation", "2001-6-22..2001-6-28", "RU-Chechen Republic",
+                            "Battle of Vedeno", "2001-8-13..2001-8-26", "RU-Chechen Insurgents", 
+                            "Operation Crescent Wind", "2001-10-7..2001-12?", "US/UK-Taliban",
+                            "Operation Rhino", "2001-10-19..2001-10-20", "US-Taliban",
+                            "Battle of Mazar-e-Sharif","2001-11-9", "US/Northern Alliance-Taliban/al-Qaeda", 
+                            "Siege of Kunduz", "2001-11-11..2001-11-23", "US/Northern Alliance-Taliban/al-Qaeda",
+                            "Battle of Herat", "Twelve of November of two thousand and twenty-one", "US/Northern Alliance/Iran-Taliban", 
+                            "Battle of Kabul", "2001-11-13..2001-11-14", "US/Northern Alliance-Taliban",
+                            "Battle of Tarin Kowt", "2001-11-13:2001-11-14", "US/Eastern Alliance-Taliban",
+                            "Operation Trent", "2001-11-~15..2001-11-~30", "US/UK-Taliban/al-Qaeda",
+                            "Battle of Kandahar", "2001-11-22..2001-12-07", "US/AU/Eastern Alliance-Taliban",
+                            "Battle of Qala-i-Jangi", "2001-11-25:2001-12-01", "US/UK/Northern Alliance-Taliban/al-Qaeda",
+                            "Battle of Tora Bora", "2001-12-12..2001-12-17", "US/Northern Alliance-Taliban/al-Qaeda",
+                            "Battle of Shawali Kowt", "2001-12-3", "US/Eastern Alliance-Taliban",
+                            "Battle of Sayyd Alma Kalay", "2001-12-4", "US/Eastern Alliance-Taliban",
+                            "Battle of Amami-Oshima", "2001-12-22", "JP-KP",
+                            "Tsotsin-Yurt operation", "2001-12-30:2002-01-03", "RU-Chechen Insurgents")
R> battles$Date = as_messydate(battles$Date)
R> tibble::tibble(battles)
\end{CodeInput}
\begin{CodeOutput}
# A tibble: 20 x 3
   Battle                               Date                     Parties        
   <chr>                                <mdate>                  <chr>          
 1 Operation MH-2                       2001-03-08               MK-National Li~
 2 2001 Bangladesh–India border clashes 2001-04-16..2001-04-20   BD-ID          
 3 Operation Vaksince                   2001-05-25               MK-National Li~
 4 Alkhan-Kala operation                2001-06-22..2001-06-28   RU-Chechen Rep~
 5 Battle of Vedeno                     2001-08-13..2001-08-26   RU-Chechen Ins~
 6 Operation Crescent Wind              2001-10-7..2001-12?      US/UK-Taliban  
 7 Operation Rhino                      2001-10-19..2001-10-20   US-Taliban     
 8 Battle of Mazar-e-Sharif             2001-11-09               US/Northern Al~
 9 Siege of Kunduz                      2001-11-11..2001-11-23   US/Northern Al~
10 Battle of Herat                      2021-11-12               US/Northern Al~
11 Battle of Kabul                      2001-11-13..2001-11-14   US/Northern Al~
12 Battle of Tarin Kowt                 2001-11-13..2001-11-14   US/Eastern All~
13 Operation Trent                      2001-11-~15..2001-11-~30 US/UK-Taliban/~
14 Battle of Kandahar                   2001-11-22..2001-12-07   US/AU/Eastern ~
15 Battle of Qala-i-Jangi               2001-11-25..2001-12-01   US/UK/Northern~
16 Battle of Tora Bora                  2001-12-12..2001-12-17   US/Northern Al~
17 Battle of Shawali Kowt               2001-12-03               US/Eastern All~
18 Battle of Sayyd Alma Kalay           2001-12-04               US/Eastern All~
19 Battle of Amami-Oshima               2001-12-22               JP-KP          
20 Tsotsin-Yurt operation               2001-12-30..2002-01-03   RU-Chechen Ins~
\end{CodeOutput}
\end{CodeChunk}

Getting the timing can be important for researchers, however, when faced
with date imprecision, researchers usually have to chose between making
arbitrary choices (e.g.~adding ``-01-01'' to all incomplete dates) or
work imprecise dates (i.e.~year only). Yet, both choices may lead to
bias results. This is especially true if researchers' are looking to
generate inferences where getting ``timing'' right is important.

Assume, for example, certain researcher is interested in the
relationship between the United States (US) being a party in a conflict
and the duration of the conflict in 2001. The researcher might theorize
that conflicts involving the US have a shorter duration since the US has
the most powerful military in the world. As well, this is relationship
could be mediated by the number of parties involved in the conflict.
That is, the number of actors involved in a conflict could have an
effect on conflict time. Using \pkg{messydates}, we can create two
different date variables in the battles data to represent conflict time,
one with an arbitrary cut of point (difference between the minimal and
maximal values) and the other with random values (difference between two
random values in the range for uncertain or approximate dates). These
will be our dependent variables. We can also create a dummy variable for
whether the US was, or not, involved in the conflict to be our main
independent variable. As a control, we also code the number of actors in
the conflict. With these variables in hand, we build two linear
regression models correlating each of the dependent variables to
conflict time.

\begin{CodeChunk}
\begin{CodeInput}
R> set.seed(3333)
R> battles <- battles %>%
+   mutate(arbitrary = as.numeric(as.Date(Date, max) - as.Date(Date, min)),
+          random = ifelse(grepl("\\?|\\~", Date),
+                          abs(as.Date(Date, random) - as.Date(Date, random)),
+                          arbitrary),
+          US_party = ifelse(grepl("US", Parties), 1, 0),
+          n_actors = c(2, 2, 2, 2, 2, 3, 2, 4, 4, 4, 3, 3, 4, 4, 5, 4, 3, 3, 2, 2))
R> arbitrary <- lm(arbitrary ~ US_party + n_actors, battles)
R> random <- lm(random ~ US_party + n_actors, battles)
R> library(stargazer)
R> stargazer::stargazer(arbitrary, random, type = "text")
\end{CodeInput}
\begin{CodeOutput}

==========================================================
                                  Dependent variable:     
                              ----------------------------
                                 arbitrary       random   
                                    (1)           (2)     
----------------------------------------------------------
US_party                          10.802         1.419    
                                 (14.399)       (6.484)   
                                                          
n_actors                          -2.479         0.670    
                                  (7.239)       (3.260)   
                                                          
Constant                           8.815         2.517    
                                 (16.241)       (7.314)   
                                                          
----------------------------------------------------------
Observations                        20             20     
R2                                 0.040         0.023    
Adjusted R2                       -0.073         -0.092   
Residual Std. Error (df = 17)     19.467         8.766    
F Statistic (df = 2; 17)           0.352         0.199    
==========================================================
Note:                          *p<0.1; **p<0.05; ***p<0.01
\end{CodeOutput}
\end{CodeChunk}

Notice how the regression coefficients change when we pick random values
within the range for the uncertain and approximate dates in the battles
data, in comparison to setting arbitrary cut off points. Although not
statistically significant, the coefficient for US being a party in a
conflict goes from being positive when we use arbitrary cut off points
for uncertain dates, to negative when we we use random values within
that range. In this case, it is hard to say whether the relationship
between the US being a part of a battle in 2001 and the time of the
conflict is positive or not.

\hypertarget{acknowledgements}{%
\section{Acknowledgements}\label{acknowledgements}}

We would like to thank the Swiss National Science Foundation. This work
was supported by grant number 188976.

\renewcommand\refname{References}
\bibliography{references.bib}



\end{document}
