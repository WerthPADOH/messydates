\documentclass[
]{jss}

\usepackage[utf8]{inputenc}

\author{
James Hollway\\Graduate Institute of\\
International and Development Studies \And Henrique Sposito\\Graduate
Institute of\\
International and Development Studies
}
\title{Working with Missing, Approximate, Uncertain, Sets and Ranges of
Dates with \pkg{messydates}}

\Plainauthor{James Hollway, Henrique Sposito}
\Plaintitle{Working with Unspecified, Approximate, Uncertain, Sets and
Ranges of Dates with messydates}
\Shorttitle{\pkg{messydates}: An R package for ISO's Extended Date/Time
Format}


\Abstract{
This paper presents the \pkg{messydates} package for R, which
facilitates working with `messy' dates. Messy dates are common when
studying historical and sometimes even current phenomena, and can create
various technical problems for the data analyst. The paper highlights
these problems and offers practical advice on how to solve them using
\pkg{messydates}. The paper also introduces a conceptual framework for
resolving messydates into more familiar date classes in R ready for
analysis.
}

\Keywords{dates, ISO, \proglang{R}}
\Plainkeywords{dates, ISO, R}

%% publication information
%% \Volume{50}
%% \Issue{9}
%% \Month{June}
%% \Year{2012}
%% \Submitdate{}
%% \Acceptdate{2012-06-04}

\Address{
    James Hollway\\
    Graduate Institute of\\
International and Development Studies\\
    Chemin Eugène-Rigot 2A\\
PO Box 1672\\
1211 Geneva 1\\
Switzerland\\
  E-mail: \email{james.hollway@graduateinstitute.ch}\\
  URL: \url{http://jameshollway.com}\\~\\
    }


% tightlist command for lists without linebreak
\providecommand{\tightlist}{%
  \setlength{\itemsep}{0pt}\setlength{\parskip}{0pt}}




\usepackage{amsmath}

\begin{document}



\hypertarget{introduction}{%
\section{Introduction}\label{introduction}}

Dates are often messy. Whether historical (or ancient), future, or even
recent, we often only know approximately when an event occurred, that it
happened within a particular period, an unreliable source means a date
should be flagged as uncertain, or sources offer multiple, competing
dates.

As researchers, we often recognise this messiness but are forced to
force non-existent precision on data so we can proceed with analysis.
For example, if we only know something happened in a given month or
year, we might just opt for the start of that month
(e.g.~\texttt{2021-07-01}) or year (\texttt{2021-01-01}), assuming that
to err on the earlier (or later) side is a justifiable bias. However,
this can create issues for inference in which sequence or timing is
important. The goal of \pkg{messydates} is to help with this problem by
retaining and working with various kinds of date imprecision.

\hypertarget{a-quick-overview}{%
\subsection{A quick overview}\label{a-quick-overview}}

\pkg{messydates} implements the extended annotation standard for dates,
the Extended Date/Time Format (EDTF), outlined in
\href{https://www.iso.org/standard/70908.html}{ISO 8601-2\_2019(E)} for
R. These include standardised notation for:

\begin{itemize}
\tightlist
\item
  unspecified date( component)s, e.g.~\texttt{2012-XX-01} for the first
  of some unknown month in 2012 or 2012-01 for some unknown day in
  January 2012
\item
  approximate date( component)s,
  e.g.~\texttt{2012-01-12\textasciitilde{}} for approximately the 12th
  of January 2012
\item
  uncertain date( component)s, e.g.~\texttt{2012-01-12?} where this data
  point is based on an unreliable source
\item
  sets of dates, e.g.~\texttt{\{2012-01-01,2012-01-12\}} where the date
  can be both 1 January 2012 and 12 January 2012
\item
  ranges of dates, e.g.~\texttt{2012-01-01..2012-01-12} for all dates
  between the 1 January 2012 and 12 January 2012 inclusive
\end{itemize}

\pkg{messydates} contains a set of tools for constructing and coercing
into and from the \texttt{messydt} class. This date class allows regular
dates to be annotated to express unspecified date components,
approximate or uncertain date components, date ranges, and sets of
dates.

Importantly, the package also includes a function for unpacking or
expanding sets or ranges of dates into all dates consistent with how the
date or set of dates is specified or annotated. Methods are also offered
that can be used to make explicit how researchers convert date
imprecision into precise dates for analysis, such as getting the
\texttt{min()}, \texttt{max()}, or even a \texttt{random()} date from
among the dates consistent with a set or range of dates. This greatly
facilitates research transparency as well as robustness checks.

\hypertarget{relationship-to-other-packages}{%
\subsection{Relationship to other
packages}\label{relationship-to-other-packages}}

\pkg{messydates} offers a new date class, but one that comes with
methods for converting from and into common date classes such as
\texttt{Date}, \texttt{POSIXct}, and \texttt{POSIXlt.} It is thus fully
compatible with packages such as \pkg{lubridate}
\citep{grolemundDatesTimesMade2011} and \pkg{anytime}
\citep{eddelbuettelAnytimeEasierDate2019}. It is also compatible with
\pkg{unstruwwel}, which also parses historic dates in R, though the
emphasis of our package is on working with these dates.

\section[R code]{\proglang{R} code}\label{r-code}

\hypertarget{a-new-class}{%
\subsection{A new class}\label{a-new-class}}

\pkg{messydates} contains a set of tools for constructing and coercing
into and from the messydt class. This date class implements ISO
8601-2:2019(E) and allows regular dates to be annotated to express
unspecified date components, approximate or uncertain date components,
date ranges, and sets of dates. The function \texttt{as\_messydate()}
handles the coercion to messydt class.

\begin{CodeChunk}
\begin{CodeInput}
R> library(messydates)
R> suppressPackageStartupMessages(library(lubridate))
R> library(tibble)
R> suppressPackageStartupMessages(library(dplyr))
R> dates_comparison <- tibble::tribble(~Example, ~OriginalDate,
+                                     "Normal date", "2010-01-01",
+                                     "Historical date", "1291-08-01",
+                                     "Very historical date", "476",
+                                     "Really historical date", "33 BC",
+                                     "Clearly future date", "9999-12-31",
+                                     "Not so clearly future date", "2599-12-31",
+                                     "Range of dates", "2019-11-01:2020-01-01",
+                                     "Uncertain date", "2001-01-01?",
+                                     "Set of dates", "2021-5-26, 2021-6-10, 2021-11-19, 2021-12-4")
R> dates_comparison %>% dplyr::mutate(base = as.Date(OriginalDate),
+                                    lubridate = suppressWarnings(lubridate::as_date(OriginalDate)),
+                                    messydates = messydates::as_messydate(OriginalDate)) %>%
+   print()
\end{CodeInput}
\begin{CodeOutput}
# A tibble: 9 x 5
  Example                    OriginalDate       base       lubridate  messydates
  <chr>                      <chr>              <date>     <date>     <messydt> 
1 Normal date                2010-01-01         2010-01-01 2010-01-01 2010-01-0~
2 Historical date            1291-08-01         1291-08-01 1291-08-01 1291-08-0~
3 Very historical date       476                NA         NA         0476     ~
4 Really historical date     33 BC              NA         NA         -0033    ~
5 Clearly future date        9999-12-31         9999-12-31 9999-12-31 9999-12-3~
6 Not so clearly future date 2599-12-31         2599-12-31 2599-12-31 2599-12-3~
7 Range of dates             2019-11-01:2020-0~ 2019-11-01 2019-11-01 2019-11-0~
8 Uncertain date             2001-01-01?        2001-01-01 2001-01-01 2001-01-0~
9 Set of dates               2021-5-26, 2021-6~ 2021-05-26 NA         2021-05-2~
\end{CodeOutput}
\end{CodeChunk}

\hypertarget{annotate}{%
\subsection{Annotate}\label{annotate}}

Some datasets have, for example, an arbitrary cut off point for start
and end points, but these are often coded as precise dates when they are
not necessarily the real start or end dates. The annotate functions
helps annotate uncertainty and approximation to dates. Inaccurate start
or end dates can be represented by an affix indicating ``on or before'',
if used as a prefix (e.g.~\texttt{..1816-01-01}), or indicating ``on or
after'', if used as a suffix (e.g.~\texttt{2016-12-31..}). Approximate
dates are indicated by adding a \texttt{\textasciitilde{}} to year,
month, or day components, as well as groups of components or whole dates
to estimate values that are possibly correct
(e.g.~\texttt{2003-03-03\textasciitilde{}}). Day, month, or year,
uncertainty can be indicated by adding a \texttt{?} to a possibly
dubious date (e.g.~\texttt{1916-10-10?}) or date component
(e.g.~\texttt{1916-?10-10}).

\begin{CodeChunk}
\begin{CodeInput}
R> dates_annotate <- tibble::tibble(Beg = as_messydate(c("1816-01-01", "1916-01-01", "2016-01-01")),
+                                  End = as_messydate(c("1816-12-31", "1916-12-31", "2016-12-31")))
R> dplyr::mutate(dates_annotate, Beg = ifelse(Beg <= "1816-01-01", on_or_before(Beg), Beg))
\end{CodeInput}
\begin{CodeOutput}
# A tibble: 3 x 2
  Beg          End       
  <chr>        <messydt> 
1 ..1816-01-01 1816-12-31
2 1916-01-01   1916-12-31
3 2016-01-01   2016-12-31
\end{CodeOutput}
\begin{CodeInput}
R> dplyr::mutate(dates_annotate, End = ifelse(End >= "2016-01-01", on_or_after(End), End))
\end{CodeInput}
\begin{CodeOutput}
# A tibble: 3 x 2
  Beg        End         
  <messydt>  <chr>       
1 1816-01-01 1816-12-31  
2 1916-01-01 1916-12-31  
3 2016-01-01 2016-12-31..
\end{CodeOutput}
\begin{CodeInput}
R> dplyr::mutate(dates_annotate, Beg = ifelse(Beg == "1916-01-01", as_approximate(Beg), Beg))
\end{CodeInput}
\begin{CodeOutput}
# A tibble: 3 x 2
  Beg         End       
  <chr>       <messydt> 
1 1816-01-01  1816-12-31
2 1916-01-01~ 1916-12-31
3 2016-01-01  2016-12-31
\end{CodeOutput}
\begin{CodeInput}
R> dplyr::mutate(dates_annotate, End = ifelse(End == "1916-12-31", as_uncertain(End), End))
\end{CodeInput}
\begin{CodeOutput}
# A tibble: 3 x 2
  Beg        End        
  <messydt>  <chr>      
1 1816-01-01 1816-12-31 
2 1916-01-01 1916-12-31?
3 2016-01-01 2016-12-31 
\end{CodeOutput}
\end{CodeChunk}

\hypertarget{expand}{%
\subsection{Expand}\label{expand}}

Expand functions transform date ranges, sets of dates, and unspecified
or approximate dates (annotated with `..', `\{ , \}', `XX' or
`\textasciitilde{}') into lists of dates. As these dates may refer to
several possible dates, the function ``opens'' these values to include
all the possible dates implied.

\begin{CodeChunk}
\begin{CodeInput}
R> dates_expand <- as_messydate(c("2008-03-25", "2001-01?", "2001",
+                                "2001-01-01..2001-02-02", "{2001-01-01,2001-02-02}",
+                                "2008-XX-31", "28 BC"))
R> expand(dates_expand)
\end{CodeInput}
\begin{CodeOutput}
[[1]]
[1] "2008-03-25"

[[2]]
 [1] "2001-01-01" "2001-01-02" "2001-01-03" "2001-01-04" "2001-01-05"
 [6] "2001-01-06" "2001-01-07" "2001-01-08" "2001-01-09" "2001-01-10"
[11] "2001-01-11" "2001-01-12" "2001-01-13" "2001-01-14" "2001-01-15"
[16] "2001-01-16" "2001-01-17" "2001-01-18" "2001-01-19" "2001-01-20"
[21] "2001-01-21" "2001-01-22" "2001-01-23" "2001-01-24" "2001-01-25"
[26] "2001-01-26" "2001-01-27" "2001-01-28" "2001-01-29" "2001-01-30"
[31] "2001-01-31"

[[3]]
  [1] "2001-01-01" "2001-01-02" "2001-01-03" "2001-01-04" "2001-01-05"
  [6] "2001-01-06" "2001-01-07" "2001-01-08" "2001-01-09" "2001-01-10"
 [11] "2001-01-11" "2001-01-12" "2001-01-13" "2001-01-14" "2001-01-15"
 [16] "2001-01-16" "2001-01-17" "2001-01-18" "2001-01-19" "2001-01-20"
 [21] "2001-01-21" "2001-01-22" "2001-01-23" "2001-01-24" "2001-01-25"
 [26] "2001-01-26" "2001-01-27" "2001-01-28" "2001-01-29" "2001-01-30"
 [31] "2001-01-31" "2001-02-01" "2001-02-02" "2001-02-03" "2001-02-04"
 [36] "2001-02-05" "2001-02-06" "2001-02-07" "2001-02-08" "2001-02-09"
 [41] "2001-02-10" "2001-02-11" "2001-02-12" "2001-02-13" "2001-02-14"
 [46] "2001-02-15" "2001-02-16" "2001-02-17" "2001-02-18" "2001-02-19"
 [51] "2001-02-20" "2001-02-21" "2001-02-22" "2001-02-23" "2001-02-24"
 [56] "2001-02-25" "2001-02-26" "2001-02-27" "2001-02-28" "2001-03-01"
 [61] "2001-03-02" "2001-03-03" "2001-03-04" "2001-03-05" "2001-03-06"
 [66] "2001-03-07" "2001-03-08" "2001-03-09" "2001-03-10" "2001-03-11"
 [71] "2001-03-12" "2001-03-13" "2001-03-14" "2001-03-15" "2001-03-16"
 [76] "2001-03-17" "2001-03-18" "2001-03-19" "2001-03-20" "2001-03-21"
 [81] "2001-03-22" "2001-03-23" "2001-03-24" "2001-03-25" "2001-03-26"
 [86] "2001-03-27" "2001-03-28" "2001-03-29" "2001-03-30" "2001-03-31"
 [91] "2001-04-01" "2001-04-02" "2001-04-03" "2001-04-04" "2001-04-05"
 [96] "2001-04-06" "2001-04-07" "2001-04-08" "2001-04-09" "2001-04-10"
[101] "2001-04-11" "2001-04-12" "2001-04-13" "2001-04-14" "2001-04-15"
[106] "2001-04-16" "2001-04-17" "2001-04-18" "2001-04-19" "2001-04-20"
[111] "2001-04-21" "2001-04-22" "2001-04-23" "2001-04-24" "2001-04-25"
[116] "2001-04-26" "2001-04-27" "2001-04-28" "2001-04-29" "2001-04-30"
[121] "2001-05-01" "2001-05-02" "2001-05-03" "2001-05-04" "2001-05-05"
[126] "2001-05-06" "2001-05-07" "2001-05-08" "2001-05-09" "2001-05-10"
[131] "2001-05-11" "2001-05-12" "2001-05-13" "2001-05-14" "2001-05-15"
[136] "2001-05-16" "2001-05-17" "2001-05-18" "2001-05-19" "2001-05-20"
[141] "2001-05-21" "2001-05-22" "2001-05-23" "2001-05-24" "2001-05-25"
[146] "2001-05-26" "2001-05-27" "2001-05-28" "2001-05-29" "2001-05-30"
[151] "2001-05-31" "2001-06-01" "2001-06-02" "2001-06-03" "2001-06-04"
[156] "2001-06-05" "2001-06-06" "2001-06-07" "2001-06-08" "2001-06-09"
[161] "2001-06-10" "2001-06-11" "2001-06-12" "2001-06-13" "2001-06-14"
[166] "2001-06-15" "2001-06-16" "2001-06-17" "2001-06-18" "2001-06-19"
[171] "2001-06-20" "2001-06-21" "2001-06-22" "2001-06-23" "2001-06-24"
[176] "2001-06-25" "2001-06-26" "2001-06-27" "2001-06-28" "2001-06-29"
[181] "2001-06-30" "2001-07-01" "2001-07-02" "2001-07-03" "2001-07-04"
[186] "2001-07-05" "2001-07-06" "2001-07-07" "2001-07-08" "2001-07-09"
[191] "2001-07-10" "2001-07-11" "2001-07-12" "2001-07-13" "2001-07-14"
[196] "2001-07-15" "2001-07-16" "2001-07-17" "2001-07-18" "2001-07-19"
[201] "2001-07-20" "2001-07-21" "2001-07-22" "2001-07-23" "2001-07-24"
[206] "2001-07-25" "2001-07-26" "2001-07-27" "2001-07-28" "2001-07-29"
[211] "2001-07-30" "2001-07-31" "2001-08-01" "2001-08-02" "2001-08-03"
[216] "2001-08-04" "2001-08-05" "2001-08-06" "2001-08-07" "2001-08-08"
[221] "2001-08-09" "2001-08-10" "2001-08-11" "2001-08-12" "2001-08-13"
[226] "2001-08-14" "2001-08-15" "2001-08-16" "2001-08-17" "2001-08-18"
[231] "2001-08-19" "2001-08-20" "2001-08-21" "2001-08-22" "2001-08-23"
[236] "2001-08-24" "2001-08-25" "2001-08-26" "2001-08-27" "2001-08-28"
[241] "2001-08-29" "2001-08-30" "2001-08-31" "2001-09-01" "2001-09-02"
[246] "2001-09-03" "2001-09-04" "2001-09-05" "2001-09-06" "2001-09-07"
[251] "2001-09-08" "2001-09-09" "2001-09-10" "2001-09-11" "2001-09-12"
[256] "2001-09-13" "2001-09-14" "2001-09-15" "2001-09-16" "2001-09-17"
[261] "2001-09-18" "2001-09-19" "2001-09-20" "2001-09-21" "2001-09-22"
[266] "2001-09-23" "2001-09-24" "2001-09-25" "2001-09-26" "2001-09-27"
[271] "2001-09-28" "2001-09-29" "2001-09-30" "2001-10-01" "2001-10-02"
[276] "2001-10-03" "2001-10-04" "2001-10-05" "2001-10-06" "2001-10-07"
[281] "2001-10-08" "2001-10-09" "2001-10-10" "2001-10-11" "2001-10-12"
[286] "2001-10-13" "2001-10-14" "2001-10-15" "2001-10-16" "2001-10-17"
[291] "2001-10-18" "2001-10-19" "2001-10-20" "2001-10-21" "2001-10-22"
[296] "2001-10-23" "2001-10-24" "2001-10-25" "2001-10-26" "2001-10-27"
[301] "2001-10-28" "2001-10-29" "2001-10-30" "2001-10-31" "2001-11-01"
[306] "2001-11-02" "2001-11-03" "2001-11-04" "2001-11-05" "2001-11-06"
[311] "2001-11-07" "2001-11-08" "2001-11-09" "2001-11-10" "2001-11-11"
[316] "2001-11-12" "2001-11-13" "2001-11-14" "2001-11-15" "2001-11-16"
[321] "2001-11-17" "2001-11-18" "2001-11-19" "2001-11-20" "2001-11-21"
[326] "2001-11-22" "2001-11-23" "2001-11-24" "2001-11-25" "2001-11-26"
[331] "2001-11-27" "2001-11-28" "2001-11-29" "2001-11-30" "2001-12-01"
[336] "2001-12-02" "2001-12-03" "2001-12-04" "2001-12-05" "2001-12-06"
[341] "2001-12-07" "2001-12-08" "2001-12-09" "2001-12-10" "2001-12-11"
[346] "2001-12-12" "2001-12-13" "2001-12-14" "2001-12-15" "2001-12-16"
[351] "2001-12-17" "2001-12-18" "2001-12-19" "2001-12-20" "2001-12-21"
[356] "2001-12-22" "2001-12-23" "2001-12-24" "2001-12-25" "2001-12-26"
[361] "2001-12-27" "2001-12-28" "2001-12-29" "2001-12-30" "2001-12-31"

[[4]]
 [1] "2001-01-01" "2001-01-02" "2001-01-03" "2001-01-04" "2001-01-05"
 [6] "2001-01-06" "2001-01-07" "2001-01-08" "2001-01-09" "2001-01-10"
[11] "2001-01-11" "2001-01-12" "2001-01-13" "2001-01-14" "2001-01-15"
[16] "2001-01-16" "2001-01-17" "2001-01-18" "2001-01-19" "2001-01-20"
[21] "2001-01-21" "2001-01-22" "2001-01-23" "2001-01-24" "2001-01-25"
[26] "2001-01-26" "2001-01-27" "2001-01-28" "2001-01-29" "2001-01-30"
[31] "2001-01-31" "2001-02-01" "2001-02-02"

[[5]]
[1] "2001-01-01" "2001-02-02"

[[6]]
 [1] "2008-01-31" "2008-02-28" "2008-03-31" "2008-04-30" "2008-05-31"
 [6] "2008-06-30" "2008-07-31" "2008-08-31" "2008-09-30" "2008-10-31"
[11] "2008-11-30" "2008-12-31"

[[7]]
  [1] "-0028-01-01" "-0028-01-02" "-0028-01-03" "-0028-01-04" "-0028-01-05"
  [6] "-0028-01-06" "-0028-01-07" "-0028-01-08" "-0028-01-09" "-0028-01-10"
 [11] "-0028-01-11" "-0028-01-12" "-0028-01-13" "-0028-01-14" "-0028-01-15"
 [16] "-0028-01-16" "-0028-01-17" "-0028-01-18" "-0028-01-19" "-0028-01-20"
 [21] "-0028-01-21" "-0028-01-22" "-0028-01-23" "-0028-01-24" "-0028-01-25"
 [26] "-0028-01-26" "-0028-01-27" "-0028-01-28" "-0028-01-29" "-0028-01-30"
 [31] "-0028-01-31" "-0028-02-01" "-0028-02-02" "-0028-02-03" "-0028-02-04"
 [36] "-0028-02-05" "-0028-02-06" "-0028-02-07" "-0028-02-08" "-0028-02-09"
 [41] "-0028-02-10" "-0028-02-11" "-0028-02-12" "-0028-02-13" "-0028-02-14"
 [46] "-0028-02-15" "-0028-02-16" "-0028-02-17" "-0028-02-18" "-0028-02-19"
 [51] "-0028-02-20" "-0028-02-21" "-0028-02-22" "-0028-02-23" "-0028-02-24"
 [56] "-0028-02-25" "-0028-02-26" "-0028-02-27" "-0028-02-28" "-0028-02-29"
 [61] "-0028-03-01" "-0028-03-02" "-0028-03-03" "-0028-03-04" "-0028-03-05"
 [66] "-0028-03-06" "-0028-03-07" "-0028-03-08" "-0028-03-09" "-0028-03-10"
 [71] "-0028-03-11" "-0028-03-12" "-0028-03-13" "-0028-03-14" "-0028-03-15"
 [76] "-0028-03-16" "-0028-03-17" "-0028-03-18" "-0028-03-19" "-0028-03-20"
 [81] "-0028-03-21" "-0028-03-22" "-0028-03-23" "-0028-03-24" "-0028-03-25"
 [86] "-0028-03-26" "-0028-03-27" "-0028-03-28" "-0028-03-29" "-0028-03-30"
 [91] "-0028-03-31" "-0028-04-01" "-0028-04-02" "-0028-04-03" "-0028-04-04"
 [96] "-0028-04-05" "-0028-04-06" "-0028-04-07" "-0028-04-08" "-0028-04-09"
[101] "-0028-04-10" "-0028-04-11" "-0028-04-12" "-0028-04-13" "-0028-04-14"
[106] "-0028-04-15" "-0028-04-16" "-0028-04-17" "-0028-04-18" "-0028-04-19"
[111] "-0028-04-20" "-0028-04-21" "-0028-04-22" "-0028-04-23" "-0028-04-24"
[116] "-0028-04-25" "-0028-04-26" "-0028-04-27" "-0028-04-28" "-0028-04-29"
[121] "-0028-04-30" "-0028-05-01" "-0028-05-02" "-0028-05-03" "-0028-05-04"
[126] "-0028-05-05" "-0028-05-06" "-0028-05-07" "-0028-05-08" "-0028-05-09"
[131] "-0028-05-10" "-0028-05-11" "-0028-05-12" "-0028-05-13" "-0028-05-14"
[136] "-0028-05-15" "-0028-05-16" "-0028-05-17" "-0028-05-18" "-0028-05-19"
[141] "-0028-05-20" "-0028-05-21" "-0028-05-22" "-0028-05-23" "-0028-05-24"
[146] "-0028-05-25" "-0028-05-26" "-0028-05-27" "-0028-05-28" "-0028-05-29"
[151] "-0028-05-30" "-0028-05-31" "-0028-06-01" "-0028-06-02" "-0028-06-03"
[156] "-0028-06-04" "-0028-06-05" "-0028-06-06" "-0028-06-07" "-0028-06-08"
[161] "-0028-06-09" "-0028-06-10" "-0028-06-11" "-0028-06-12" "-0028-06-13"
[166] "-0028-06-14" "-0028-06-15" "-0028-06-16" "-0028-06-17" "-0028-06-18"
[171] "-0028-06-19" "-0028-06-20" "-0028-06-21" "-0028-06-22" "-0028-06-23"
[176] "-0028-06-24" "-0028-06-25" "-0028-06-26" "-0028-06-27" "-0028-06-28"
[181] "-0028-06-29" "-0028-06-30" "-0028-07-01" "-0028-07-02" "-0028-07-03"
[186] "-0028-07-04" "-0028-07-05" "-0028-07-06" "-0028-07-07" "-0028-07-08"
[191] "-0028-07-09" "-0028-07-10" "-0028-07-11" "-0028-07-12" "-0028-07-13"
[196] "-0028-07-14" "-0028-07-15" "-0028-07-16" "-0028-07-17" "-0028-07-18"
[201] "-0028-07-19" "-0028-07-20" "-0028-07-21" "-0028-07-22" "-0028-07-23"
[206] "-0028-07-24" "-0028-07-25" "-0028-07-26" "-0028-07-27" "-0028-07-28"
[211] "-0028-07-29" "-0028-07-30" "-0028-07-31" "-0028-08-01" "-0028-08-02"
[216] "-0028-08-03" "-0028-08-04" "-0028-08-05" "-0028-08-06" "-0028-08-07"
[221] "-0028-08-08" "-0028-08-09" "-0028-08-10" "-0028-08-11" "-0028-08-12"
[226] "-0028-08-13" "-0028-08-14" "-0028-08-15" "-0028-08-16" "-0028-08-17"
[231] "-0028-08-18" "-0028-08-19" "-0028-08-20" "-0028-08-21" "-0028-08-22"
[236] "-0028-08-23" "-0028-08-24" "-0028-08-25" "-0028-08-26" "-0028-08-27"
[241] "-0028-08-28" "-0028-08-29" "-0028-08-30" "-0028-08-31" "-0028-09-01"
[246] "-0028-09-02" "-0028-09-03" "-0028-09-04" "-0028-09-05" "-0028-09-06"
[251] "-0028-09-07" "-0028-09-08" "-0028-09-09" "-0028-09-10" "-0028-09-11"
[256] "-0028-09-12" "-0028-09-13" "-0028-09-14" "-0028-09-15" "-0028-09-16"
[261] "-0028-09-17" "-0028-09-18" "-0028-09-19" "-0028-09-20" "-0028-09-21"
[266] "-0028-09-22" "-0028-09-23" "-0028-09-24" "-0028-09-25" "-0028-09-26"
[271] "-0028-09-27" "-0028-09-28" "-0028-09-29" "-0028-09-30" "-0028-10-01"
[276] "-0028-10-02" "-0028-10-03" "-0028-10-04" "-0028-10-05" "-0028-10-06"
[281] "-0028-10-07" "-0028-10-08" "-0028-10-09" "-0028-10-10" "-0028-10-11"
[286] "-0028-10-12" "-0028-10-13" "-0028-10-14" "-0028-10-15" "-0028-10-16"
[291] "-0028-10-17" "-0028-10-18" "-0028-10-19" "-0028-10-20" "-0028-10-21"
[296] "-0028-10-22" "-0028-10-23" "-0028-10-24" "-0028-10-25" "-0028-10-26"
[301] "-0028-10-27" "-0028-10-28" "-0028-10-29" "-0028-10-30" "-0028-10-31"
[306] "-0028-11-01" "-0028-11-02" "-0028-11-03" "-0028-11-04" "-0028-11-05"
[311] "-0028-11-06" "-0028-11-07" "-0028-11-08" "-0028-11-09" "-0028-11-10"
[316] "-0028-11-11" "-0028-11-12" "-0028-11-13" "-0028-11-14" "-0028-11-15"
[321] "-0028-11-16" "-0028-11-17" "-0028-11-18" "-0028-11-19" "-0028-11-20"
[326] "-0028-11-21" "-0028-11-22" "-0028-11-23" "-0028-11-24" "-0028-11-25"
[331] "-0028-11-26" "-0028-11-27" "-0028-11-28" "-0028-11-29" "-0028-11-30"
[336] "-0028-12-01" "-0028-12-02" "-0028-12-03" "-0028-12-04" "-0028-12-05"
[341] "-0028-12-06" "-0028-12-07" "-0028-12-08" "-0028-12-09" "-0028-12-10"
[346] "-0028-12-11" "-0028-12-12" "-0028-12-13" "-0028-12-14" "-0028-12-15"
[351] "-0028-12-16" "-0028-12-17" "-0028-12-18" "-0028-12-19" "-0028-12-20"
[356] "-0028-12-21" "-0028-12-22" "-0028-12-23" "-0028-12-24" "-0028-12-25"
[361] "-0028-12-26" "-0028-12-27" "-0028-12-28" "-0028-12-29" "-0028-12-30"
[366] "-0028-12-31"
\end{CodeOutput}
\end{CodeChunk}

\hypertarget{contract}{%
\subsection{Contract}\label{contract}}

The \texttt{contract()} function operates as the opposite of
\texttt{expand()}. It contracts a list of dates into the abbreviated
annotation of \pkg{messydates}.

\begin{CodeChunk}
\begin{CodeInput}
R> tibble::tibble(contract = contract(expand(dates_expand)))
\end{CodeInput}
\begin{CodeOutput}
# A tibble: 7 x 1
  contract                                                                      
  <messydt>                                                                     
1 2008-03-25                                                                   ~
2 2001-01                                                                      ~
3 2001                                                                         ~
4 2001-01-01..2001-02-02                                                       ~
5 {2001-01-01,2001-02-02}                                                      ~
6 {2008-01-31,2008-02-28,2008-03-31,2008-04-30,2008-05-31,2008-06-30,2008-07-31~
7 -0028                                                                        ~
\end{CodeOutput}
\end{CodeChunk}

\hypertarget{coerce-from-messydates}{%
\subsection{Coerce from messydates}\label{coerce-from-messydates}}

Coercion functions coerce objects of \texttt{messydt} class to common
date classes such as \texttt{Date}, \texttt{POSIXct}, and
\texttt{POSIXlt}. Since \texttt{messydt} objects can hold multiple
individual dates, an additional function must be passed as an argument
so that multiple dates are ``resolved'' into a single date.

For example, one might wish to use the earliest possible date in any
ranges of dates (\texttt{min}), the latest possible date (\texttt{max}),
some notion of a central tendency (\texttt{mean}, \texttt{median}, or
\texttt{modal}), or even a \texttt{random} selection from amongst the
candidate dates.

These functions are particularly useful for use with existing methods
and models, especially for checking the robustness of results.

\begin{CodeChunk}
\begin{CodeInput}
R> tibble::tibble(min = as.Date(dates_expand, min),
+                max = as.Date(dates_expand, max),
+                median = as.Date(dates_expand, median),
+                mean = as.Date(dates_expand, mean),
+                modal = as.Date(dates_expand, modal),
+                random = as.Date(dates_expand, random))
\end{CodeInput}
\begin{CodeOutput}
# A tibble: 7 x 6
  min        max        median     mean       modal      random    
  <date>     <date>     <date>     <date>     <date>     <date>    
1 2008-03-25 2008-03-25 2008-03-25 2008-03-25 2008-03-25 2008-03-25
2 2001-01-01 2001-01-31 2001-01-16 2001-01-16 2001-01-01 2001-01-06
3 2001-01-01 2001-12-31 2001-07-02 2001-07-02 2001-01-01 2001-03-05
4 2001-01-01 2001-02-02 2001-01-17 2001-01-17 2001-01-01 2001-01-30
5 2001-01-01 2001-02-02 2001-02-02 2001-01-17 2001-01-01 2001-01-01
6 2008-01-31 2008-12-31 2008-07-31 2008-07-15 2008-01-31 2008-07-31
7 -028-01-01 -028-12-31 -028-07-02 -028-07-01 -028-01-01 -028-10-31
\end{CodeOutput}
\end{CodeChunk}

\hypertarget{additional-functionality}{%
\subsection{Additional functionality}\label{additional-functionality}}

Several other functions are also offered in the \pkg{messydates}
package.

For example, one can check various logical tests for messy date objects.
\texttt{is\_messydate()} tests whether the object inherits the
\texttt{messydt} class. \texttt{is\_intersecting()} tests whether there
is any intersection between two messy dates. \texttt{is\_element()}
similarly tests whether a messy date can be found within a messy date
range or set. \texttt{is\_similar()} tests whether two dates contain
similar components.

\begin{CodeChunk}
\begin{CodeInput}
R> is_messydate(as_messydate("2012-01-01"))
\end{CodeInput}
\begin{CodeOutput}
[1] TRUE
\end{CodeOutput}
\begin{CodeInput}
R> is_messydate(as.Date("2012-01-01"))
\end{CodeInput}
\begin{CodeOutput}
[1] FALSE
\end{CodeOutput}
\begin{CodeInput}
R> is_intersecting(as_messydate("2012-01"), as_messydate("2012-01-01..2012-02-22"))
\end{CodeInput}
\begin{CodeOutput}
[1] FALSE
\end{CodeOutput}
\begin{CodeInput}
R> is_intersecting(as_messydate("2012-01"), as_messydate("2012-02-01..2012-02-22"))
\end{CodeInput}
\begin{CodeOutput}
[1] FALSE
\end{CodeOutput}
\begin{CodeInput}
R> is_element(as_messydate("2012-01-01"), as_messydate("2012-01"))
\end{CodeInput}
\begin{CodeOutput}
[1] TRUE
\end{CodeOutput}
\begin{CodeInput}
R> is_element(as_messydate("2012-01-01"), as_messydate("2012-02"))
\end{CodeInput}
\begin{CodeOutput}
[1] FALSE
\end{CodeOutput}
\begin{CodeInput}
R> is_similar(as_messydate("2012-06-02"), as_messydate("2012-02-06"))
\end{CodeInput}
\begin{CodeOutput}
[1] TRUE
\end{CodeOutput}
\begin{CodeInput}
R> is_similar(as_messydate("2012-06-22"), as_messydate("2012-02-06"))
\end{CodeInput}
\begin{CodeOutput}
[1] FALSE
\end{CodeOutput}
\end{CodeChunk}

Additionally, one can perform intersection (\texttt{md\_intersect()})
and union (\texttt{md\_union()}) on, inter alia, messy date class
objects. Or `join' that retains all elements, even if duplicated, with
\texttt{md\_multiset}.

\begin{CodeChunk}
\begin{CodeInput}
R> md_intersect(as_messydate("2012-01-01..2012-01-20"),as_messydate("2012-01"))
\end{CodeInput}
\begin{CodeOutput}
 [1] "2012-01-01" "2012-01-02" "2012-01-03" "2012-01-04" "2012-01-05"
 [6] "2012-01-06" "2012-01-07" "2012-01-08" "2012-01-09" "2012-01-10"
[11] "2012-01-11" "2012-01-12" "2012-01-13" "2012-01-14" "2012-01-15"
[16] "2012-01-16" "2012-01-17" "2012-01-18" "2012-01-19" "2012-01-20"
\end{CodeOutput}
\begin{CodeInput}
R> md_union(as_messydate("2012-01-01..2012-01-20"),as_messydate("2012-01"))
\end{CodeInput}
\begin{CodeOutput}
 [1] "2012-01-01" "2012-01-02" "2012-01-03" "2012-01-04" "2012-01-05"
 [6] "2012-01-06" "2012-01-07" "2012-01-08" "2012-01-09" "2012-01-10"
[11] "2012-01-11" "2012-01-12" "2012-01-13" "2012-01-14" "2012-01-15"
[16] "2012-01-16" "2012-01-17" "2012-01-18" "2012-01-19" "2012-01-20"
[21] "2012-01-21" "2012-01-22" "2012-01-23" "2012-01-24" "2012-01-25"
[26] "2012-01-26" "2012-01-27" "2012-01-28" "2012-01-29" "2012-01-30"
[31] "2012-01-31"
\end{CodeOutput}
\begin{CodeInput}
R> md_multiset(as_messydate("2012-01-01..2012-01-20"),as_messydate("2012-01"))
\end{CodeInput}
\begin{CodeOutput}
 [1] "2012-01-01" "2012-01-02" "2012-01-03" "2012-01-04" "2012-01-05"
 [6] "2012-01-06" "2012-01-07" "2012-01-08" "2012-01-09" "2012-01-10"
[11] "2012-01-11" "2012-01-12" "2012-01-13" "2012-01-14" "2012-01-15"
[16] "2012-01-16" "2012-01-17" "2012-01-18" "2012-01-19" "2012-01-20"
[21] "2012-01-01" "2012-01-02" "2012-01-03" "2012-01-04" "2012-01-05"
[26] "2012-01-06" "2012-01-07" "2012-01-08" "2012-01-09" "2012-01-10"
[31] "2012-01-11" "2012-01-12" "2012-01-13" "2012-01-14" "2012-01-15"
[36] "2012-01-16" "2012-01-17" "2012-01-18" "2012-01-19" "2012-01-20"
[41] "2012-01-21" "2012-01-22" "2012-01-23" "2012-01-24" "2012-01-25"
[46] "2012-01-26" "2012-01-27" "2012-01-28" "2012-01-29" "2012-01-30"
[51] "2012-01-31"
\end{CodeOutput}
\end{CodeChunk}

\hypertarget{acknowledgements}{%
\section{Acknowledgements}\label{acknowledgements}}

We would like to thank the Swiss National Science Foundation. This work
was supported by grant number 188976.

\renewcommand\refname{References}
\bibliography{references.bib}



\end{document}
